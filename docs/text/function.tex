\section{Functionality description}
This section goes over the app functionalities and how they are designed and implemented.

\subsection{Login}
This is a section going over logging in to the application. If it is the first time the user uses the application they will have to create an account with username
and password. Otherwise only password is required. If user inserts password that does not match their corresponding password stored in database the 
"Invalid password" message is displayed. If the password is correct, the main window of the application is shown.

This is implemented in the file \texttt{main.py} in the function \texttt{start\_app()}.

\subsection{File selection}
Selecting files is handled by the button \texttt{Choose a file to be sent}. When the button is pressed, a dialog window is opened where the user can select
a file. The implementation of this is done using module \texttt{filedialog} from \texttt{tkinter} library. If a file is not selected, the window wont let them
select only path that is not pointing to a file. If the window is closed without selecting the filepath is kept empty.

The code itself can be seen in file \texttt{file\_share/app/app.py} in the function \texttt{get\_file()}.

\subsection{Listing users}
There are two functionalities that are related to listing users. The first one is listing all non-friend users that are in the database. The second one 
is listing all users that user has added as a friends. Both of these open a new windows where the users are listed. The difference between these two windows
is that in window where non-friend users are listed, there is a button to add them as a friend.

The list friends functionality is simpler and it is done by retrieving all users from the database and filtering out the ones that are friends with the user.
The method that handles this is the \texttt{get\_all\_users} from \texttt{file\_share/database/\_\_init.py\_\_}, where there is friend parameter which is by default set to \texttt{True}.\\
The non-friends is a bit more complicated first the list of all users except friends is collected from the database. It is listed in the window. If user wants 
to add a friend, he selects that user from the list and pressed button "Befriend this user", this calls a function \texttt{befriend} in \texttt{file\_share/database/\_\_init.py\_\_}.
This adds the user as a friend by setting the \texttt{is\_friend} attribute to \texttt{True}.

\subsection{File sending initialization}
For this core functionality there is attached a flow chart for easier understandability. It is implemented in function \texttt{send\_file} in \texttt{file\_share/app/app.py}.
The function first checks if the file for transfer is selected, if that is so then the file is prepared for transfer and file transfer is attepmted sending is done by 
function \texttt{send\_or\_store\_file} from \texttt{file\_share/sender/sender.py}. As a return there is a \texttt{SendStatus} object, which is enum created for this application,
there are 5 states that the object can be in, those can be viewed in \texttt{file\_share/definitions/enums.py}. Based on the the state of the user which is passed to the function,
the file will either be sent, stored in outgoing queue or nothing will happen.\\

If the file is to be stored then it is scheduled to be added to files table in the database using \texttt{store\_file} from \texttt{file\_share/database/\_\_init.py\_\_}.

\begin{figure}[ht]
    \centering
    \includegraphics[scale = 0.5]{images/filesending.png}
    \caption{Flowchart of file sending initialization}
    \label{fig:flowchart}
\end{figure}
\subsection{File sending}
